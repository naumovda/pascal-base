\documentclass{beamer}
\mode<presentation>
\usetheme{CambridgeUS}
\usepackage[russian]{babel}
\usepackage[utf8]{inputenc}
\usepackage[T2A]{fontenc}
\usepackage{sansmathaccent}

\usepackage{verbatim}
\usepackage{alltt}

\pdfmapfile{+sansmathaccent.map}
\title[Модули]{Модули}
\author{Наумов Д.А., доц. каф. КТ, ИТГД }
\date[01.04.2020] {Программирование и алгоритмические языки, 2020}

\begin{document}

%ТИТУЛЬНЫЙ СЛАЙД
\begin{frame}
  \titlepage
\end{frame}
  
%СОДЕРЖАНИЕ ЛЕКЦИИ
\begin{frame}
  \frametitle{Содержание лекции}
  \tableofcontents  
\end{frame}
  
%РАЗДЕЛ 1
\section{Модули}

\begin{frame}
\begin{block}{Модульное программирование}
организация программы как совокупности независимых блоков (модулей).
\end{block}
Использование модульного программирования:
\begin{enumerate}
\item упрощает разработку и тестирование программу;
\item дает возможность разрабатывать программу группе разработчиков;
\item позволяет проводить обновление (замену) модуля без изменения остальной части системы.
\end{enumerate}
Основной принцип структурного программирования: декомпозиция сложных на более простые подззадачи.
\end{frame} 

\begin{frame}
\begin{block}{Модуль}
автономно компилируемая программная единица, реализующая определенную функциональность
и предоставляющая интерфейс к ней.
\end{block}
В Pascal модуль представляющая собой библиотеку типов данных, констант, переменных, процедур и функций.
\begin{enumerate}
\item исходный текст программы: pas
\item результат компиляции программы: exe
\item исходный текст модуля: pas
\item результат компиляции модуля: tpu (Turbo pascal), dcu (Delphi), pcu (Pascal ABC).
\end{enumerate}
\end{frame} 

\begin{frame}[fragile]{Структура модуля}
\begin{alltt}
 1 unit <Имя модуля>;//Заголовок модуля 	
 2 //Интерфейсная секция (открытые описания)
 3   interface 
 4	    uses <СписокМодулей>; 
 5	    <РазделОткрытыхОписаний>
 6 //Секция реализации (закрытые описания)
 7   implementation 
 8	    uses <СписокМодулей>;  
 9	    <РазделЗакрытыхОписаний>
10 //Секция инициализациия
11   begin
12	    <РазделИнициализации>
13   end.
\end{alltt}
Заголовок модуля является обязательным, имя модуля должно быть идентификатором языка Pascal и совпадать с именем файла модуля.
\end{frame}

\begin{frame}[fragile]
Для связи модулей друг с другом используется секция uses, где перечисляются имена подключаемых модулей.
\begin{enumerate}
\item порядок подключения модулей в современных реализациях языка Pascal не важен;
\item повторное подключение модуля в одной программе (модуле) приведет к ошибке компиляции.
\end{enumerate}
\begin{alltt}
//текст модуля
1 unit Complex; //заголовок модуля
2 end. 

//текст программы
1 uses Complex; //подключаем модуль
2 begin 
3 end.
\end{alltt}
\end{frame} 

\begin{frame}[fragile]{Интерфейсная секция}
\begin{block}{Интерфейсная секция}
содержит описания, которые будут доступны в других модулях и программах при подключении данного модуля.
\end{block}
Интерфейсная секция может содержать:
\begin{itemize}
\item раздел подключения модулей (чтобы использовать их в разделе описания данной секции); 
\item раздел описания открытых констант, переменных, типов данных;
\item описание (сигнатуры, заголовки) открытых процедур и функций.
\end{itemize}
\end{frame}

\begin{frame}[fragile]{Фрагмент модуля работы с комплексными числами}
\begin{alltt}
 1 unit Complex;
 2 interface
 3   type
 4     TСElem = real;
 5     TComplex = record Re, Im: TСElem; end;
 6   const
 7     C_ZERO: TComplex = (Re:0; Im:0);
 8     C_ONE: TComplex = (Re:1; Im:0);
 9     C_IM_ONE: TComplex = (Re:0; Im:1);
 10  procedure Init(var A: TComplex; Re, Im: TСElem);
 11  procedure Add(const A, B: TComplex; var C: TComplex);
 12  procedure Sub(const A, B: TComplex; var C: TComplex); 
 13  function GetModule(const A: TComplex): TCElem;
 14  function GetArgument(const A: TComplex): TCElem;
\end{alltt}
\end{frame} 

\begin{frame}[fragile]{Cекция реализации}
\begin{block}{Cекция реализации должна содержать:}
\begin{itemize}
\item определение (реализация) процедур и функций, описанных в интерфейсной секции;
\end{itemize}
\end{block}
При определении процедур и функций, описанных в интерфейсной части, можно опускать список формальных параметров.
\begin{block}{Cекция реализации может содержать:}
\begin{itemize}
\item раздел подключения модулей (чтобы использовать их в данной секции и секции инициализации); 
\item раздел описания закрытых констант, переменных, типов данных;
\item определение (реализация) закрытых процедур и функций;
\end{itemize}
\end{block}
Константы, переменные, типы данных, подпрограммы, описанные в секции реализации,
являются локальными для модуля и не будут видны в программах и модулях, которые подключат данный модуль.
\end{frame}

\begin{frame}[fragile]{Фрагмент модуля работы с комплексными числами}
\begin{alltt}
15 implementation
16   const
17     EPS = 1e-6;
18   var
19     iCError: integer;
20   procedure Init(Re, Im: TСElem; var A: TComplex);
21	 begin
22      A.Re := Re; A.Im := Im;
23	 end;
24   procedure Add(const A, B: TComplex; var C: TComplex);
25   begin
26      C.Re := A.Re + B.Re;
27      C.Im := A.Im + B.Im;
28   end;
\end{alltt}
\end{frame} 

\begin{frame}[fragile]{Cекция инициализации}
\begin{block}{В секции инициализации}
размещаются операторы, которые исполняются один раз в процессе запуска программы и
используются для подготовки работы модуля (инициализация значений переменных и т.д.).
\end{block}
Запуск секций инициализации осуществляется в порядке следования имен модулей в секции uses.
\begin{alltt}
29 begin
30   iCError := 0;
31 end;
\end{alltt}
Пример модуля: ComplexUnit.pas.

Пример использования модуля: ComplexProgram.pas.
\end{frame}

\end{document}
